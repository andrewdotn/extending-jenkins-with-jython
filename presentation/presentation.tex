\documentclass[xcolor=svgnames,17pt]{beamer}

\usepackage[export]{adjustbox}
\usepackage{bookmark}
\usepackage{colortbl} \arrayrulecolor[gray]{0.7}
%\usepackage{lmodern}
\usepackage{microtype}
\usepackage{pgfpages}
\usepackage{rotating}
\usepackage{textcomp}
\usepackage{tabularx}
\usepackage{xspace}

\usepackage{fontspec}

\hypersetup{hidelinks,pdfpagemode=}

\urlstyle{same}

\newcommand*{\sizefont}[1]{%
    \ifcase#1\relax
    \or \tiny
    \or \scriptsize
    \or \footnotesize
    \or \small
    \or \normalsize
    \or \large
    \or \Large
    \or \LARGE
    \or \huge
    \or \Huge
    \fi}

%%

\newcommand*{\mybullet}{\tikz[baseline=-.6ex]\node[%
    draw,circle,inner sep = -0.15ex,fill]{.};\xspace}

\setbeamertemplate{footline}{
    \usebeamercolor[fg]{page number in head/foot}%
    \usebeamerfont{page number in head/foot}%
    \hspace*{1ex}\insertframenumber\,/\,\inserttotalframenumber\hfill
    Extending Jenkins with Jython}

\newcommand*{\plainfooter}{%
    \setbeamertemplate{footline}{
        \usebeamercolor[fg]{page number in head/foot}%
        \usebeamerfont{page number in head/foot}%
        \hspace*{1ex}\insertframenumber\,/\,\inserttotalframenumber\vskip2pt}}

\makeatletter
\def\alphslide{\@alph{\intcalcAdd{1}{\intcalcSub{\thepage}{\beamer@framestartpage}}}}
\newcommand*{\plainstepfooter}{
    \setbeamertemplate{footline}{
        \usebeamercolor[fg]{page number in head/foot}%
        \usebeamerfont{page number in head/foot}%
        \hspace*{1ex}\insertframenumber\alphslide\,/\,\inserttotalframenumber\vskip2pt}}
\makeatother

\setbeamertemplate{note page}{
    \sizefont{3}
    \setlength{\parskip}{10pt}
    \insertnote
    \par}

\setbeamertemplate{navigation symbols}{}
\setbeamerfont{title}{size=\LARGE}
\setbeamerfont{frametitle}{size=\LARGE}
\setbeamerfont{framesubtitle}{size=\normalsize}

%% \AtBeginSection{\frame{\tableofcontents[current]}}

\newcommand*{\tocsection}[1]{\pdfbookmark[2]{#1}{#1}}

%%

\title{Extending Jenkins \\ with Jython}

\author{\texorpdfstring{%
    Andrew Neitsch}{Andrew Neitsch}}

\date{\small 2015-10-23}

\begin{document}

\tocsection{Title page}

\frame[plain]{\titlepage}

\note{This is a note}

\begin{frame}{Outline}
\tableofcontents
\end{frame}

\section{Introduction}

\begin{frame}{Continuous Integration}
\end{frame}

\note{Blah}

\section{Test Isolation}

\begin{frame}{}
\tableofcontents[currentsection]
\end{frame}

\section{Extending Jenkins}

\begin{frame}{}
\tableofcontents[currentsection]
\end{frame}

\begin{frame}{Plugins}
\end{frame}

\section{Conclusions}

\begin{frame}{}
\tableofcontents[currentsection]
\end{frame}

\begin{frame}
\adjustbox{width=0.5\paperwidth,center}{\structure{Questions?}}
\end{frame}

\begin{frame}{Call for talks}

“w/o interesting talks, there's not a ton of point in ‘meeting up’”

\pause

\begin{enumerate}
\item Pick something you find interesting
\item Talk about it
\item Include exercises for people to hack on
\end{enumerate}

\end{frame}

\begin{frame}{Suggested exercises}
\begin{itemize}
\item Install Jenkins, and get it testing \\ some code you care about
\item Isolate your tests
\item Try extending Jenkins
\end{itemize}
\end{frame}

\end{document}
